\section{Introduction}
\label{sec:introduction}

\nemquote{%
From the ashes a fire shall be woken,\\
A light from the shadows shall spring;\\
Renewed shall be blade that was broken,\\
The crownless again shall be king.
}{J.R.R. Tolkien}

\nemchapterfirstletter{T}{rustless}, high-performance, layered-architecture, blockchain-based DLT protocol - these are the first principles that influenced the development of \codename.
While other DLT protocols were considered, including DAG and dBFT, blockchain was quickly chosen as the protocol most true to the ideal of trustlessness.
Any node can download a complete copy of the chain and can independently verify it at all times.
Nodes with sufficient harvesting power can always create blocks and never need to rely on a leader.
These choices necessarily sacrifice some throughput relative to other protocols, but they seem most consistent with the philosophical underpinnings of Bitcoin\cite{nakamoto2009}.

As part of a focus on trustlessness, the following features were added relative to NEM:
\begin{itemize}
	\item{Block headers can be synced without transaction data, while allowing verification of chain integrity.}
	\item{Transaction Merkle trees allow cryptographic proofs of transaction containment (or not) within blocks.}
	\item{Receipts increase the transparency of indirectly triggered state changes.}
	\item{State proofs allow trustless verification of specific state within the blockchain.}
\end{itemize}

In \codename, there is a single server executable that can be customized by loading different plugins (for transaction support) and extensions (for functionality).
There are three primary configurations (per network), but further customized hybrid configurations are possible by enabling or disabling specific extensions.

The three primary configurations are:
\begin{enumerate}
	\item{Peer: These nodes are the backbone of the network and create new blocks.}
	\item{API: These nodes store data in a MongoDB database for easier querying and can be used in combination with a NodeJS REST server.}
	\item{Dual: These nodes perform the functions of both Peer and API nodes.}
\end{enumerate}

A strong network will typically have a large number of Peer nodes and enough API nodes to support incoming client requests.
Allowing the composition of nodes to vary dynamically based on real needs, should lead to a more globally resource-optimized network.

Basing the core block and transaction pipelines on the disruptor pattern - and using parallel processing wherever possible - allows high rates of transactions per second relative to a typical blockchain protocol.

NIS1 was a worthy entry into the blockchain landscape and \codenamespace is an even worthier evolution of it.
This is not the end but a new beginning.
There is more work to be done.

\subsection{Variants}

\codenamespace supports compile-time substitution of its primary hash algorithm, which produces a 32-byte value from input data.
For conciseness, the rest of this document refers to this hash by its default setting (SHA3).
Currently, the following hash algorithms are supported:

\begin{enumerate}
	\item{SHA3 (default): This is the default mode and recommended for new chains.}
	\item{Keccak: This mode is provided for compatibility with legacy chains, like NIS1, in order to preserve private-key:public-key:address mapping.}
\end{enumerate}
