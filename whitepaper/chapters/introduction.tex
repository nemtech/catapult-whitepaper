\section{Introduction}
\label{sec:introduction}

\nemquote{%
From the ashes a fire shall be woken,\\
A light from the shadows shall spring;\\
Renewed shall be blade that was broken,\\
The crownless again shall be king.
}{J.R.R. Tolkien}

\nemchapterfirstletter{T}{rustless}, high-performance, layered-architecture, blockchain-based DLT protocol - these are the first principles that influenced the development of \codename.
While a handful of DLT protocols were considered, a blockchain protocol was quickly chosen because it was deemed most true to the ideal of trustlessness.
Any node can download a complete copy of the chain and can independently verify it at all times.
Nodes with sufficient harvesting power can always create blocks and never need to rely on a leader.
These choices necessarily sacrifice some throughput relative to other protocols, but they seem most consistent with the philosophical underpinnings of Bitcoin\cite{nakamoto2009}.

\codenamespace supports both probabilistic and deterministic block finalization.
Under probabilistic finalization, the probability that any particular block is rolled back decreases as more and more blocks are added - or chained - to it.
Although the probability might become vanishingly small, it is always nonzero\footnote{
	When probabilistic finalization is enabled, for performance reasons, at most \nemsetting{network}{maxRollbackBlocks} are allowed to be rolled back.
	Older blocks are assumed, but not guaranteed, to be finalized because the probability of their rollback is quite low.
}.
In contrast, deterministic finalization includes a mechanism in the protocol that allows checkpoints to be set that can never be rolled back.
This can lead to potentially deep rollbacks but provides stronger assurances.
In either case, the risk of block rollback and transaction invalidation is assumed by users.

As part of a focus on trustlessness, the following features were added relative to NEM:
\begin{itemize}
	\item{Block headers can be synced without transaction data, while allowing verification of chain integrity.}
	\item{Transaction Merkle trees allow cryptographic proofs of transaction containment (or not) within blocks.}
	\item{Receipts increase the transparency of indirectly triggered state changes.}
	\item{State proofs allow trustless verification of specific state within the blockchain.}
\end{itemize}

In \codename, there is a single server executable that can be customized by loading different plugins (for transaction support) and extensions (for functionality).
There are three primary configurations (per network), but further customized hybrid configurations are possible by enabling or disabling specific extensions.

The three primary configurations are:
\begin{enumerate}
	\item{Peer: These nodes are the backbone of the network and create new blocks.}
	\item{API: These nodes store data in a MongoDB database for easier querying and can be used in combination with a NodeJS REST server.}
	\item{Dual: These nodes perform the functions of both Peer and API nodes.}
\end{enumerate}

A strong network will typically have a large number of Peer nodes and enough API nodes to support incoming client requests.
Allowing the composition of nodes to vary dynamically based on real needs, should lead to a more globally resource-optimized network.

Basing the core block and transaction pipelines on the disruptor pattern - and using parallel processing wherever possible - allows high rates of transactions per second relative to a typical blockchain protocol.

NIS1 was a worthy entry into the blockchain landscape and \codenamespace is an even worthier evolution of it.
This is not the end but a new beginning.
There is more work to be done.

\subsection{Network Fingerprint}

Each network has a unique \nind{fingerprint} that is composed of the following:
\begin{enumerate}
	\item{Network Identifier -
		One byte identifier that can be shared across networks.
		All addresses compatible with a network must have this value as their first byte.
	}
	\item{Generation Hash Seed -
		Random 32 byte value that must be globally unique across networks.
		This value is prepended to transaction data prior to hashing or signing in order to prevent cross network replay attacks.
	}
\end{enumerate}
