\section{System}
\label{sec:system}

\nemquote{%
A new technology can transform society, but when the technology is in its infancy, very few people can see its full potential.
}{Liu Cixin}

\codenamechapterfirstword supports high customization at both the network and individual node levels.
Network-wide settings, specified in \nemsetting{network}{network}, must be the same for all nodes in a network.
In contrast, node-specific settings can vary across all nodes in the same network, and are located in \nemsetting{node}{node}.

\codenamespace was designed to use a \emph{plugin / extension} approach instead of supporting Turing complete smart contracts.
While the latter can allow for more user flexibility, it's also more error prone from the user perspective.
A plugin model limits the operations that can be performed on a blockchain and consequently has a smaller attack surface.
Additionally, it's much simplier to optimize the performance of a discrete set of operations than an infinite set.
This helps \codenamespace achieve the high throughput for which it was designed.

\subsection{Transaction Plugins}

All nodes in a network must support the same types of transactions and process them in exactly the same manner so that all nodes can agree on the global blockchain state.
The network requires each node to load a set of \emph{transaction plugins}\index{plugins}, and this set determines the kinds of transactions the network supports.
This set is determined by the presence of \nemsetting{network}{plugin*} sections in the \nemsetting{network}{network} configuration.
Any changes, additions or deletions of these plugins must be coordinated and accepted by all network nodes.
If only a subset of nodes agree to these modifications, those nodes will be on a fork.
All built-in \codenamespace transactions are built using this plugin model in order to validate its extensibility.

A plugin is a separate dynamically-linked library that exposes a single entry point in the following form
\footnote{Format of plugins depends on target operating system and compiler used, so all host applications and plugins must be built with the same compiler version and options.}:

\begin{lstlisting}
	extern "C" PLUGIN_API
	void RegisterSubsystem(
		catapult::plugins::PluginManager& manager);
\end{lstlisting}

\class{PluginManager} provides access to the subset of configuration that plugins need to initialize themselves.
Through this class, a plugin can register zero or more of the following:

\begin{enumerate}
	\item{Transactions -
		New transaction types and the mapping of those types to parsing rules can be specified.
		Specifically, the plugin defines rules for translating a transaction into component notifications that are used in further processing.
		A handful of processing constraints can also be specified, such as indicating a transaction can only appear within an aggregate transaction \nemrefparens{sec:transactions:aggregate}.
	}
	\item{Caches -
		New cache types and rules for serializing and deserializing model types to and from binary can be specified.
		Each state-related cache can optionally be included in the calculation of a block's \field{StateHash} \nemrefparens{sec:blocks:statehash} when that feature is enabled.
	}
	\item{Handlers - These are APIs that are always accessible.}
	\item{Diagnostics - These are APIs and counters that are accessible only when the node is running in diagnostic mode.}
	\item{Validators -
		Stateless and stateful validators process the notifications produced by block and transaction processing.
		The registered validators can subscribe to general or plugin-defined notifications and reject disallowed values or state changes.
	}
	\item{Observers -
		Observers process the notifications produced by block and transaction processing.
		The registered observers can subscribe to general or plugin-defined notifications and update blockchain state based on their values.
		Observers don't require any validation logic because they are only called after all applicable validators succeed.
	}
	\item{Resolvers -
		Custom mappings from unresolved to resolved types can be specified.
		For example, this is used by the namespace plugin to add support for alias resolutions.
	}
\end{enumerate}

\subsection{\codenamespace Extensions}
\label{sec:system:extensions}

Individual nodes within a network are allowed to support a heterogeneous mix of capabilities.
For example, some nodes might want to store data in a database or publish events to a message queue.
These capabilities are all optional because none of them impact consensus.
Such capabilities are determined by the set of \nind{extensions} a node loads as specified in \nemsetting{extensions-\{process\}}{extensions}.
Most built-in \codenamespace functionality is built using this extension model in order to validate its extensibility.

An extension is a separate dynamically linked library that exposes a single entry point in the following form
\footnote{Format of extensions depends on target operating system and compiler used, so all host applications and plugins must be built with the same compiler version and options.}:

\begin{lstlisting}
	extern "C" PLUGIN_API
	void RegisterExtension(
		catapult::extensions::ProcessBootstrapper& bootstrapper);
\end{lstlisting}

\class{ProcessBootstrapper} provides access to full \codenamespace configuration and services that extensions need to initialize themselves.
Providing this additional access allows extensions to be more powerful than plugins.
Through this class, an extension can register zero or more of the following:

\begin{enumerate}
	\item{Services -
		A service represents an independent behavior.
		Services are passed an object representing the executable's state and can use it to configure a multitude of things.
		Among others, a service can add diagnostic counters, define APIs (both diagnostic and non-diagnostic) and add tasks to the task scheduler.
		It can also create dependent services and tie their lifetimes to that of the hosting executable.
		There are very few limitations on what a service can do, which allows the potential for significant customizations.
	}
	\item{Subscriptions -
		An extension can subscribe to any supported blockchain event.
		Events are raised when changes are detected.
		Block, state, unconfirmed transaction and partial transaction change events are supported.
		Transaction status events are raised when the processing of a transaction completes.
		Node events are raised when remote nodes are discovered.
	}
\end{enumerate}

In addition to the above, extensions can configure the node in more intricate ways.
For example, an extension can register a custom network time supplier.
In fact, there is a specialized extension that sets a time supplier based on algorithms described in \nemref{sec:timesync}.
This is an example of the high levels of customization allowed by this extension model.
For understanding the full range of extensibility allowed by extensions, please refer to the project code or developer documentation\nemtechdocsfootnote{}.

\subsection{Server}

The simplest \codenamespace topology is composed of a single server executable.
Transaction Plugins required by the network and \codenamespace Extensions desired by the node operator are loaded and initialized by the server.

\codenamespace stores all of its data in a \filepath{data} directory.
The contents of the data directory are as follows:

\begin{enumerate}
	\item{Block Versioned Directories -
		These directories contain block information in a proprietary format.
		Each confirmed block's binary data, transactions and associated data are stored in these directories.
		The statements \nemrefparens{sec:blocks:receipts} generated when processing each block are also stored here for quick access.
		An example of a versioned directory is \filepath{00000}, which contains the first group of blocks.
	}
	\item{\filepath{audit} - Audit files created by the audit consumer \nemrefparens{sec:disruptor:commonConsumers} are stored in this directory.}
	\item{\filepath{spool} -
		Subscription notifications are written out to this directory.
		They are used as a message queue to pass messages from the server to the broker.
		They are also used by the recovery process to recover data in the case of an ungraceful termination.
	}
	\item{\filepath{state} -
		\codenamespace stores its proprietary storage files in this directory.
		\filepath{supplemental.dat} and files ending with \filepath{\_summary.dat} store summarized data.
		Files ending in \filepath{Cache.dat} store complete cache data.
	}
	\item{\filepath{statedb} - When \nemsetting{node}{enableCacheDatabaseStorage} is set, this directory will contain RocksDB files.}
	\item{\filepath{transfer\_message} -
		When \nemsetting{user}{enableDelegatedHarvestersAutoDetection} is set, this directory will contain extracted delegated harvesting requests for the current node.
	}
	\item{\filepath{commit\_step.dat} -
		This stores the most recent step of the commit process initiated by the server.
		This is primarily used for recovery purposes.
	}
	\item{\filepath{index.dat} - This is a counter that contains the number of blocks stored on disk.}
\end{enumerate}

\subsubsection{Cache Database}

The server can run with or without a cache database.
When \nemsetting{node}{enableCacheDatabaseStorage} is set, RocksDB is used to store cache data.
Verifiable state \nemrefparens{sec:blocks:statehash} requires a cache database and most network configurations are expected to have it enabled.

A cache database should only be \emph{disabled} when all of the following are true:

\begin{enumerate}
	\item{High rate of transactions per second is desired.}
	\item{Trustless verification of cache state is not important.}
	\item{Servers are configured with a large amount of RAM.}
\end{enumerate}

In this mode, all cache entries are always resident in memory.
On shutdown, cache data is written to disk across multiple flat files.
On startup, this data is read and used to populate the memory caches.

When a cache database is enabled, summarized cache data is written to disk across multiple flat files.
This summarized data is derivable from data stored in caches.
One example is the list all high-value accounts that have a balance of at least \nemsetting{network}{minHarvesterBalance}.
While this list can be generated by (re)inspecting all accounts stored in the account state cache, it is saved to and loaded from disk as an optimization.

\subsection{Broker}

The broker process allows more complex \codenamespace behaviors to be added without sacrificing parallelization.
Transaction Plugins required by the network and \codenamespace Extensions desired by the node operator are loaded and initialized by the broker.
Although the broker supports all features of Transaction Plugins, it only supports a subset of \codenamespace Extensions features.
For example, overriding the network time supplier in the broker is not supported.
Broker extensions are primarily intended to register subscribers and react to events forwarded to those subscribers.
Accordingly, it's expected that the server and broker have different extensions loaded.
Please refer to the project code or developer documentation for more details.

The broker monitors the \filepath{spool} directories for changes and forwards any event notifications to subscribers registered by loaded extensions.
Extensions register their subscribers to process these events.
For example, a database extension can read these events and use them to update a database to accurately reflect the global blockchain state.

\filepath{spool} directories function as one way message queues.
The server writes messages and the broker reads them.
There is no way for the broker to send messages to the server.
This decoupling is intentional and was done for performance reasons.

The server raises subscription events in the blockchain sync consumer \nemrefparens{sec:disruptor:blockConsumers} when it holds an exclusive lock to the blockchain data.
These operations are offloaded to the broker to prevent slow database operations when the server has an exclusive lock.
The server overhead is minimal because most of the data used by the broker is also required to recover data after an ungraceful server termination.

\subsection{Recovery}

The recovery process is used to repair the global blockchain state after an ungraceful server and/or broker termination.
Transaction Plugins required by the network and \codenamespace Extensions desired by the node operator are loaded and initialized by the recovery process.
When a broker is used, the recovery process must load the same extensions as the broker.

The specific recovery procedure depends on the process configuration and the value of the \filepath{commit\_step.dat} file.
Generally, if the server exited after state changes were flushed to disk, those changes will be reapplied.
The blockchain state will be the same as if the server had applied and committed those changes.
Otherwise, if the server exited before state changes were flushed to disk, pending changes will be discarded.
The blockchain state will be the same as if the server had never attempted to process those changes.

After the recovery process completes, the blockchain state should be indistinguishable from the state of a node that never terminated ungracefully.
\filepath{spool} directories are repaired and processed.
Block and cache data stored on disk are reconciled and updated.
Pending state changes, if applicable, are applied.
Other files indicating the presence of an ungraceful termination are updated or removed.

\subsection{Common Topologies}

Although a network can be composed of a large number of heterogeneous topologies, it is likely that most nodes will fall into one of three categories: Peer, API or Dual.
The same server process is used across all of these topologies.
The only difference is in what extensions each loads.

Peer nodes are lightweight nodes.
They have enough functionality to add security to the blockchain network, but little beyond that.
They can synchronize with other nodes and harvest new blocks.

API nodes are more heavyweight nodes.
They can synchronize with other nodes, but cannot harvest new blocks.
They support hosting bonded aggregate transactions and collecting cosignatures to complete them.
They require a broker process, which is configured to write data into a MongoDB database and propagate changes over public message queues to subscribers.
The REST API is dependent on both of these capabilities and is typically co-located with an API node for performance reasons in order to minimize latency.

Dual nodes are simply a superset of Peer and API nodes.
They support all capabilities of both node types.
Since these nodes support all API node capabilities, they also require a broker.
