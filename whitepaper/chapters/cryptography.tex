\section{Cryptography}
\label{sec:cryptography}

\nemquote{%
I understood the importance in principle of public key cryptography but it's all moved much faster than I expected. I did not expect it to be a mainstay of advanced communications technology.
}{Whitfield Diffie}

\nemchapterfirstletter{B}{lockchain} technology demands the use of some cryptographic concepts.
\codenamespace uses cryptography based on Elliptic Curve Cryptography (ECC).
The choice of the underlying curve is important in order to guarantee security and speed.

\codenamespace uses the Ed25519\index{Ed25519} digital signature algorithm.
This algorithm uses the following \emph{Twisted Edwards curve}:
$$ -x^2 + y^2 = 1 - \frac{121665}{121666} x^2 y^2$$
over the finite field defined by the prime number $2^{255}-19$.
The base point for the corresponding group G is called B.
The group has $q=2^{252} + 27742317777372353535851937790883648493$ elements.
It was developed by D. J. Bernstein et al. and is one of the safest and fastest digital signature algorithms \cite{Bernstein2011}.
For the hash function $H$ mentioned in the paper, \codenamespace uses the 512-bit SHA3 hash function.

Importantly for \codenamespace purposes, the algorithm produces short 64-byte signatures and supports fast signature verification.
Neither key generation nor signing is used during block processing, so the speed of these operations is unimportant.

\subsection{Public/Private Key Pair}

A \nind{private key} is a random 256-bit integer $k$. To derive the public key\index{public key} \underline{A} from it, the following steps are taken:
\begin{align}
	\hf(k) &=(h_0, h_1,..., h_{511}) \\
	a &= 2^{254} + \sum_{3\leq i \leq 253} 2^i h_i \\
	A &= aB
\end{align}

Since $A$ is a group element, it can be encoded into a 256-bit integer \underline{A}, which serves as the public key.

\subsection{Signing and Verification}

Given a message $M$, private key $k$ and its associated public key \underline{A}, the following steps are taken to create a signature:
\begin{align}
	\hf(k) &=(h_0, h_1,..., h_{511}) \\
	r &= \hf(h_{256},...,h_{511}, M) \text{ where the comma means concatenation} \\
	R &= rB \\
	S &= (r + \hf(\underline{R}, \underline{A}, M)a) \: mod \: q \label{eq:cryptography:S}
\end{align}

Then $(\underline{R}, \underline{S})$ is the \nind{signature} for the message $M$ under the private key $k$.
Note that only signatures where $S<q$ and $S>0$ are considered as valid \textbf{to prevent} the problem of \emph{signature malleability}\index{signature!malleability}.

To verify the signature $(\underline{R}, \underline{S})$ for the given message $M$ and public key \underline{A}, the verifier checks $S<q$ and $S>0$ and then calculates
\begin{align*}
	\tilde{R} = SB - \hf(\underline{R}, \underline{A}, M)A
\end{align*}

and verifies that
\begin{equation}
	\tilde{R} = R \label{eq:cryptography:verifyR}
\end{equation}

If $S$ was computed as shown in \eqref{eq:cryptography:S} then
$$SB = rB + (\hf(\underline{R}, \underline{A}, M)a)B = R + \hf(\underline{R}, \underline{A}, M)A$$
so \eqref{eq:cryptography:verifyR} will hold.

\subsection{Batch Verification}

When lots of signatures have to be verified, a batch signature verification can speed up the process by about 80\%.
\codenamespace uses the algorithm outlined in \cite{Bernstein2011}.
Given a batch of $(M_i, A_i, R_i, S_i)$ where $(R_i, S_i)$ is the signature for the message $M_i$ with public key $A_i$,
let $H_i = \hf(R_i, A_i, M_i)$.
Additionally, assume a corresponding number of uniform distributed 128-bit independent random integers $z_i$ are generated.
Now consider the equation
\begin{align}
	\left(-\sum_i{z_i S_i \: \mathrm{mod} \: q}\right)B + \sum_i{z_i R_i} + \sum_i{(z_i H_i \: \mathrm{mod} \: q)A_i = 0} \label{eq:cryptography:verifyBatch}
\end{align}

Setting $P_i = 8 R_i + 8 H_i A_i - 8 S_i B$, then if \eqref{eq:cryptography:verifyBatch} holds, it implies
\begin{align}
	\sum_i{z_i P_i} = 0 \label{eq:cryptography:verifyBatch2}
\end{align}

All $P_i$ are elements of a cyclic group (remember $q$ is a prime).
If some $P_i$ is not zero, for example $P_2$, it means that for given integers $z_0, z_1, z_3, z_4 ...$, there is exactly one choice for $z_2$ to satisfy \eqref{eq:cryptography:verifyBatch2}.
The chance for that is $2^{-128}$.
Therefore, if \eqref{eq:cryptography:verifyBatch} holds, it is a near certainty that $P_i = 0$ for all $i$.
This implies that the signatures are valid.

If \eqref{eq:cryptography:verifyBatch} does not hold, it means that there is at least one invalid signature.
In that case, \codenamespace falls back to single signature verification to identify the invalid signatures.
